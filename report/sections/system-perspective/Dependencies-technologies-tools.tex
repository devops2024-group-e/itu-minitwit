
\subsection{Dependencies, Technologies and Tools}

% All dependencies of your *ITU-MiniTwit* systems on all levels of abstraction and development stages. That is, list and briefly describe all technologies and tools you applied and depend on.

In Table \ref{tab:tool} we briefly describe the tools and technologies used. A comprehensive list of tools containing the full reasoning for their use, can be found in appendix \ref{appendix:decision-log}.

\begin{longtable}{|p{0.3\textwidth}|p{0.2\textwidth} | p{0.4\textwidth}|}
    \hline
    \textbf{Purpose} & \textbf{Technology/Tool} & \textbf{Description}\\
    \hline
    Work log & \href{https://www.notion.so/help/guides/category/documentation}{Notion} & Notion is a web hosted workspace, that allows users to collaborate in real-time on managing projects, notes, and other types of documents in a markdown format.\\
    \hline
    Remote Repository, work tracking, and coordination & \href{https://github.com/features}{GitHub} & GitHub serves as a remote Git repository. It comes with collaborative features, such as having work items (known as issues), project coordination boards, CI/CD workflows, etc. We use GitHub, as the group has experience with the tool and it provides the collaborative features we need.\\
    \hline
    Programming language & \href{https://learn.microsoft.com/en-us/dotnet/csharp/}{C\#} &
    C\# is an object-oriented programming (OOP) language. We chose C\# due to its widespread use in the industry, making it very well documented. Furthermore, the familiarity of OOP paradigms among developers served as a supplementary factor.\\
    \hline
    Micro Web-Framework & \href{https://learn.microsoft.com/en-us/aspnet/core/razor-pages/?view=aspnetcore-8.0&tabs=visual-studio}{Razor Pages} & Razor Pages is a simple web application programming model provided by ASP.NET. It uses C\# and simple markup syntax to build web pages. Furthermore, Razor Pages offers simple configuration and an abundance of documentation.\\
    \hline
    Database Connection & \href{https://learn.microsoft.com/en-us/ef/core/}{Entity Framework} & Entity Framework (EF) serves as an object-relational mapper, which allows developers to work with databases using .NET objects. It also eliminates the need for data-access code by using LINQ to interact with the data.\\
    \hline
    Server host & \href{https://docs.digitalocean.com/}{Digital Ocean} & Digital Ocean (DO) is a cloud hosting service, which provides server hosting, database hosting, and a command line interface tool (doctl), that can be used to automatically create and drop droplets (VMs). We have chosen DO primarily due to the great documentation and online resources.\\
    \hline
    Database & \href{https://www.postgresql.org/docs/}{PostgreSQL} & PostgreSQL is an open source relational database management system known for its extensibility, and adherence to SQL standards. Additionally, the majority of developers in the team have experience with PostgreSQL.\\
    \hline
    CI/CD & \href{https://docs.github.com/en/actions}{GitHub Actions} & GitHub Actions is a CI/CD service, that allows the user to automate tasks such as building, testing, and deploying code under certain conditions. We chose GitHub Actions due to its integration with GitHub, where we store our code repository. Furthermore, it collaborates well with other tools used in the project, such as Digital Ocean.\\
    \hline
    Monitoring and Logging Graphics & \href{https://grafana.com/}{Grafana} & Grafana is a open source monitoring toolbox, that includes a variety of features such as visualising monitoring metrics and logging via customisable dashboards.\\
    \hline
    Metrics collector & \href{https://prometheus.io/docs/}{Prometheus} & Prometheus is a monitoring and alerting toolkit, that scrapes and stores data from a specific application, alerting developers of abnormalities in the system. The only requirement we had for a metrics collector was for it to collaborate with Grafana.\\
    \hline
    Metrics collector & \href{https://opentelemetry.io/docs/languages/}{OpenTelemetry} & OpenTelemetry is a metrics collector, that supports a variety of different APIs and SDKs such as .NET. We chose OpenTelemetry since it is widely used, and it is well integrated with .NET applications.\\
    \hline
    Quality of Code Analysis & \href{https://www.sonarsource.com/products/sonarcloud/}{SonarCloud and Code Climate} & SonarCloud is a cloud based code analysis service, which can be integrated into a GitHub repository, in order to analyse its code. Code Climate incorporates static analysis and test coverage as well.\\
    \hline
    Linter & \href{https://pre-commit.com/}{Pre-commit} & Pre-commit is a framework for managing pre-commit hooks. This can be used for automatically fixing code formatting and removing debug statements. We chose this tool as multiple actions can be integrated into one hook.\\
    \hline
    Linter & \href{https://github.com/hadolint/hadolint}{Hadolint} & Hadolint is an open source Dockerfile linter, that enforces best practices when building Docker images. As we have several Dockerfiles, we found it fitting to have a Dockerfile linter. Hadolint seems to be the more popular option as it is Haskell-based.\\
    \hline
    Formatter & \href{https://learn.microsoft.com/en-us/dotnet/core/tools/dotnet-format}{Dotnet-format} & Dotnet-format is a formatting tool, included in the .NET 6 SDK and onwards. It applies style preferences to a project which can be configured in an \texttt{.editorconfig} file. We chose dotnet-format since it enforces C\# coding conventions.\\
    \hline
    Logging & \href{https://grafana.com/docs/loki/latest/}{Grafana Loki} & Loki is a log aggregation system designed to store and query logs from an application, inspired by Prometheus. In order to implement a logging stack, we were interested in using a tool that was compatible with Grafana. Loki ensures this compatibility, as it is from the Grafana toolbox.\\
    \hline
    Infrastructure as code & \href{https://www.pulumi.com/docs/}{Pulumi} & Pulumi is an open source infrastructure as code SDK, that enables creation, deployment and management of infrastructure for a cloud service. We picked Pulumi, as it does not introduce a new programming language, it is open source, and it works well with CI/CD tools.\\
    \hline
    Configuration management & \href{https://docs.ansible.com/}{Ansible} & Ansible is an open source automation software used to configure and deploy systems using Ansible playbooks written in \texttt{.yaml} files. We chose Ansible as it provides an easy setup and does not require a new client to be installed on our application servers. Furthermore, the configurations is described in YAML, thereby not introducing new languages.\\
    \hline
    Scaling & \href{https://docs.docker.com/reference/cli/docker/swarm/}{Docker Swarm} & Docker Swarm is a container orchestration tool used for clustering Docker containers. Due to the team's limited experience with managing scaled systems, we chose Docker Swarm since we were already using Docker.\\
    \hline
    Dependency monitoring & \href{https://docs.github.com/en/code-security/dependabot}{Dependabot} & Dependabot is a GitHub feature used for detecting outdated packages, dependencies and actions in repositories. We chose Dependabot as it would integrate well with our CI/CD pipeline in GitHub Actions. \\
    \hline
    \caption{Tools and dependencies used in Minitwit}
    \label{tab:tool}
\end{longtable}
