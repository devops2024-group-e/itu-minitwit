\subsection{Dependencies, Technologies and tools}

% All dependencies of your *ITU-MiniTwit* systems on all levels of abstraction and development stages. That is, list and briefly describe all technologies and tools you applied and depend on.

A comprehensive list of tools containing the full reasoning for their use, can be found in appendix \ref{appendix:decision-log}.\\


\begin{longtable}{|p{0.3\textwidth}|p{0.2\textwidth} | p{0.4\textwidth}|}
    \hline
         Server host & \href{https://docs.digitalocean.com/}{Digital Ocean} & Digital Ocean (DO) is a cloud hosting service, which provides server hosting, database hosting, and a command line interface tool (doctl), which can be used in to automatically create and drop droplets (VMs).\\
         &&\\
         && We have chosen DO primarily due to the fact that it was recommended by the lecturer, but kept using digital ocean, due to the great documentation and online resources.\\
         \hline
         Database & \href{https://www.postgresql.org/docs/}{PostgreSQL} & PostgreSQL is an open-source relational database management system known for its extensibility, and adherence to SQL standarts.\\
         &&\\
         && Since the majority of developers in the team have previous experience with PostgreSQL, we chose to migrate from SQLite to PostgreSQL.\\
         \hline
         Database migration & \href{https://www.digitalocean.com/products/managed-databases}{Digital Ocean} & Digital Ocean is a cloud hosting service that provides a database hosting service, which is well documented.\\ % TODO
         \hline
         CI/CD & \href{https://docs.github.com/en/actions}{Github Actions} & Github actions is a CI/CD service, that allows you to automate tasks such as building, testing, and deploying your code when certain conditions are met. Github Actions is well documented, and it is quite seamless to configure secrets and refer to them in a .yaml file.\\
         &&\\
         && Due to it's integration with Github, we chose Github Actions, as we already use Github to store our code repository. Furthermore, it collaborates well with other tools used in the project such as Digital Ocean.\\
         \hline
         Monitoring & \href{https://grafana.com/}{Grafana} & Grafana is an open source monitoring service that can visualise monitoring metrics. This is done by configuring dashboards with charts and graphs, which visualise your system's performance.\\
         &&\\
         && Initially, we opted for \href{https://opentelemetry.io/}{OpenTelemetry} since it is a widely used metrics exporter and collector. However, due to difficulties with the setup, we switched to Grafana.\\ %TODO
         \hline
    Metrics collector & \href{https://prometheus.io/docs/}{Prometheus} & Prometheus is a monitoring and alerting toolkit, that scrapes and stores data from a specific application, alerting developers of abnormalities in the system.\\
    &&\\
    && The only requirement we had for a metrics collector was: It had to collaborate with Grafana. Furthermore, this tool was recommended by our lecturer.\\
    \hline
    Quality of Code Analysis & \href{https://www.sonarsource.com/products/sonarcloud/}{SonarCloud} & SonarCloud is a cloud based code analysis service, that can \\
    \hline
    Linter & \href{https://pre-commit.com/}{Pre-commit} & Pre-Commit is a framework for managing pre-commit hooks. This can be used for automatically fixing code format and removing debug statements.\\
    &&\\
     && 
         \hline
         Linter & \href{https://github.com/hadolint/hadolint}{Hadolint} & \\
         \hline
         Linter & \href{https://learn.microsoft.com/en-us/dotnet/core/tools/dotnet-format}{Dotnet-Format} & Dotnet-Format is a formatting tool, which is included in the .NET 6 SDK and onwards. It applies style preferences to a project which can configured in an .editorconfig file.\\
         &&\\
         && We chose Dotnet-Format since this linter enforces C\# coding conventions, and the primary programming language used in this project is C\#.\\
         \hline
         Logging & \href{https://grafana.com/}{Grafana} & \\
         \hline
         Logging & \href{https://grafana.com/docs/loki/latest/}{Grafana Loki} & \\
         \hline
         Logging & Promtail & \\
         \hline
         Provisioning & Vagrant & \\
         \hline
         Configuration management & Ansible & \\
         \hline
         Infrastructure as code & Pulumi & \\
         \hline
         Scaling & Docker swarm & \\
         \hline
         Update strategy & Rolling updates & \\
         \hline
\end{longtable}
