
\subsection{Dependencies, Technologies and tools}

% All dependencies of your *ITU-MiniTwit* systems on all levels of abstraction and development stages. That is, list and briefly describe all technologies and tools you applied and depend on.

A comprehensive list of tools containing the full reasoning for their use, can be found in appendix \ref{appendix:decision-log}

\begin{longtable}{|p{0.3\textwidth}|p{0.2\textwidth} | p{0.4\textwidth}|}
    \hline
    \textbf{Purpose} & \textbf{Technology/Tool} & \textbf{Description}\\
    \hline
    Work log & \href{https://www.notion.so/help/guides/category/documentation}{Notion} & Notion is a web hosted workspace, that allows users to collaboratively manage projects, notes, and other types of documents in a markdown format in real-time.\\
    &&\\
    && Initially, we opted to maintain our work log on Github wikis due to it's integration with the repository. However, this approach proved cumbersome as the process of making pull requests and seeking reviews for minor log updates became overly complex. Consequently, we transitioned to Notion.\\
    \hline
    Programming language & \href{https://learn.microsoft.com/en-us/dotnet/csharp/}{C\#} &
    C\#  is an object-oriented programming language, that supports development for a variety of application types such as web-application.\\
    && \\
    && For refactorisation of the application we chose C\# , due to it's widespread use in the industry, making it very well documented. Furthermore, the familiarity of object-oriented programming paradigms among developers served as a supplementary factor in the decision\\
    \hline
    Micro Web-Framework & \href{https://learn.microsoft.com/en-us/aspnet/core/razor-pages/?view=aspnetcore-8.0&tabs=visual-studio}{Razor Pages} & Razor pages is a simple web application programming model provided by ASP.NET. It uses C\# and simple markup syntax in order to build web pages, and is generally the recommended framework for page-focused, serverside web applications in .NET core.\\
    && \\
    && Initially, we attempted using Scriban, due to it's superior efficiency in terms of speed and memory usage, and it's compatibility with html templates. However, due to the lack of online resources, we switched to Razor Pages, which offered simpler configuration and an abundance of documentation, but required more code rebasing.\\
    \hline
    Database Connection & \href{https://learn.microsoft.com/en-us/ef/core/}{Entity Framework} & Entity Framework serves as a object-relational mapper, which allows developers to work with databases using .NET objects, and eliminates the need for data-access code, by using LINQ in order to interact with the data. Furthermore, EF Core supports a plethora of database engines, such as PostgreSQL and SQLite.\\
    &&\\
    && We opted for Entity Framework due to the reasons stated above.\\
    \hline
    Server host & \href{https://docs.digitalocean.com/}{Digital Ocean} & Digital Ocean (DO) is a cloud hosting service, which provides server hosting, database hosting, and a command line interface tool (doctl), which can be used in to automatically create and drop droplets (VMs).\\
    &&\\
    && We have chosen DO primarily due to the fact that it was recommended by the lecturer, but kept using digital ocean, due to the great documentation and online resources.\\
    \hline
    Database & \href{https://www.postgresql.org/docs/}{PostgreSQL} & PostgreSQL is an open-source relational database management system known for its extensibility, and adherence to SQL standards.\\
    &&\\
    && Since the majority of developers in the team have previous experience with PostgreSQL, we chose to migrate from SQLite to PostgreSQL.\\
    \hline
    Database migration & \href{https://www.digitalocean.com/products/managed-databases}{Digital Ocean} & Digital Ocean is a cloud hosting service that provides a database hosting service, which is well documented.\\ % TODO
    \hline
    CI/CD & \href{https://docs.github.com/en/actions}{Github Actions} & Github actions is a CI/CD service, that allows you to automate tasks such as building, testing, and deploying your code when certain conditions are met. Github Actions is well documented, and it is quite seamless to configure secrets and refer to them in a .yaml file.\\
    &&\\
    && Due to it's integration with Github, we chose Github Actions, as we already use Github to store our code repository. Furthermore, it collaborates well with other tools used in the project such as Digital Ocean.\\
    \hline
    Monitoring Graphics & \href{https://grafana.com/}{Grafana} & Grafana is an open source monitoring toolbox, that includes a variety of features such as visualising monitoring metrics and logging. This is done by configuring dashboards with charts and graphs, which visualise your system's performance and logging outputs.\\
    &&\\
    && Initially, we opted for \href{https://opentelemetry.io/}{OpenTelemetry} since it is a widely used metrics exporter and collector. However, due to difficulties with the setup, we switched to Grafana.\\ %TODO
    \hline
    Metrics collector & \href{https://prometheus.io/docs/}{Prometheus} & Prometheus is a monitoring and alerting toolkit, that scrapes and stores data from a specific application, alerting developers of abnormalities in the system.\\
    &&\\
    && The only requirement we had for a metrics collector was: It had to collaborate with Grafana. Furthermore, this tool was recommended by our lecturer.\\
    \hline
    Metrics collector & \href{https://opentelemetry.io/docs/languages/}{OpenTelemetry} & OpenTelemetry is a metrics collector, that supports \\
    &&\\
    && We chose OpenTelemetry since it is widely used, and it is well integrated with .NET applications.\\
    \hline
    Quality of Code Analysis & \href{https://www.sonarsource.com/products/sonarcloud/}{SonarCloud} & SonarCloud is a cloud based code analysis service, which can be integrated into a GitHub repository, in order to analyse the code in said repository.\\ %TODO
    &&\\
    && We added this software as it was an exercise in the course.\\
    \hline
    Linter & \href{https://pre-commit.com/}{Pre-commit} & Pre-Commit is a framework for managing pre-commit hooks. This can be used for automatically fixing code format and removing debug statements.\\
    &&\\
    && We chose this tool as multiple actions can be integrated into one hook. As we were tasked with integrating 3 linters into the system.\\
    \hline
    Linter & \href{https://github.com/hadolint/hadolint}{Hadolint} & Hadolint is an open-source Dockerfile linter, that enforces best practices when building docker images.\\
    &&\\
    && Due to the nature of the server setup, we have quite a few Dockerfiles. Therefore, we found it fitting to have a Dockerfile linter. From the alternative presented (\href{https://github.com/RedCoolBeans/dockerlint}{Dockerlinter}) Hadolint seems to be the more popular one as it is haskell based rather than node.js based.\\
    \hline
    Linter & \href{https://learn.microsoft.com/en-us/dotnet/core/tools/dotnet-format}{Dotnet-Format} & Dotnet-Format is a formatting tool, which is included in the .NET 6 SDK and onwards. It applies style preferences to a project which can configured in an .editorconfig file.\\
    &&\\
    && We chose Dotnet-Format since this linter enforces C\# coding conventions, and the primary programming language used in this project is C\#.\\
    \hline
    Logging & \href{https://grafana.com/}{Grafana} & See section on Monitoring Graphics.\\
    &&\\
    && We chose to use Grafana for logging as we already use Grafana for monitoring. Furthermore, we were interested in keeping the array of tools used in this project from expanding too much.\\
    \hline
    Logging & \href{https://grafana.com/docs/loki/latest/}{Grafana Loki} & Loki is a log aggregation system designed to store and query logs from an application inspired by Prometheus.\\
    &&\\
    && In order to implement a logging stack we were interested in using a tool that was compatible with Grafana. Loki ensures this compatibility, as it is from the Grafana toolbox.\\
    \hline
    Infrastructure as code & \href{https://www.pulumi.com/docs/}{Pulumi} & Pulumi is an open-source infrastructure as code SDK, that enables creation, deployment and management of infrastructure for a cloud service.\\
    &&\\
    &&  We wanted to move away from using Vagrant to deploy as it lacks collaborative features, making development difficult. We considered 3 alternatives to Vagrant: Terraform, Pulumi, and OpenTofu. In the End we picked Pulumi, as it doesn't introduce a new programming language, it is open-source, and it works well with CI/CD tools.\\
    \hline
    Configuration management & \href{https://docs.ansible.com/}{Ansible} & Ansible is an open-source automation software used to configure and deploy systems using Ansible playbooks written in .yaml files.\\
    &&\\
    && We chose Ansible as it provides an easy setup and does not require a new client to be installed on our application servers. Furthermore the configurations is described in yaml thereby not introducing new languages.\\
    \hline
    Scaling & \href{https://docs.docker.com/reference/cli/docker/swarm/}{Docker Swarm} & Docker Swarm is a container orchestration tool used for clustering Docker containers.\\
    &&\\
    && We chose Docker Swarm as the most notable alternative is Kubernetes. Due to the teams limited experience with managing scaled systems, we chose not to use Kubernetes as it is known for being harder to implement than Docker Swarm.\\
    \hline
    Update strategy & \href{https://docs.docker.com/engine/swarm/swarm-tutorial/rolling-upate/}{Rolling updates} & Rolling updates is an update strategy, that allows a system to update with 0 downtime.\\
    &&\\
    && Rolling updates is the default update strategy in Docker Swarm and therefore the one we have chosen, as to not make the implementation more complicated than it needs to be.\\
    \hline
    \caption{Tools and dependencies used in minitwit}
\end{longtable}
