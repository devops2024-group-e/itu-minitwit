\section{Security Assessment}
\subsection{Risk Identification}
\subsubsection{Asset Identification}
By browsing through our setup and documentation, we identified the following list of assets:
\begin{itemize}
    \item Web application
    \item Public GitHub repository
    \item Digital Ocean servers
    \item Tools
    \begin{itemize}
        \item Ansible
        \item Code Climate
        \item Docker, Docker Compose \& Docker Swarm
        \item GitHub Actions
        \item Grafana
        \item Linters
        \item Loki
        \item Pulumi
        \item Sonar Cloud
    \end{itemize}
\end{itemize}
\subsubsection{Threat Source Identification}
To help us identify possible threats to the system, we have consulted the OWASP \textit{Top 10 Web Application Security Risks}\cite*{OWASP}, which describes the following threats:
\begin{enumerate}
    \item Broken Access Control
    \item Cryptographic failures
    \item Injection attacks
    \item Insecure Design
    \item Security misconfiguration
    \item Vulnerable and outdated components
    \item Identification and authentication failures
    \item Software and Data integrity failures
    \item Security Logging and Monitoring Failures
    \item Server Side Request Forgery
\end{enumerate}
\subsubsection{Risk Scenario Construction}
Based on the information gathered from the two previous steps, we have constructed the following risk scenarios and outlined which of the OWASP top 10 risks affecting the scenario:
\begin{enumerate}
    \item \textbf{\underline{URL Tampering}}

    The attacker can construct URL in the \textbf{web application} with user ID such that they bypass login and are able to write message from another user’s account.

    This would be an issue of \textit{Broken Access Control} and \textit{Server Side Request Forgery}.
    \item \textbf{\underline{Log Injection}}

    The attacker can fabricate log information via injection attack in the \textbf{web application} as a means to hide their activity and ill-intentioned actions.

    This would be an issue of \textit{Security Logging and Monitoring Failures} and \textit{Injection attacks}.
    \item \textbf{\underline{Password Brute Forcing}}

    The attacker can brute force login credentials in the \textbf{web application} by taking advantage of no timeouts and weaker hash implementation.

    This would be an issue of \textit{Cryptographic failures} and \textit{Identification and authentication failures}.
    \item \textbf{\underline{Depricated Dependencies}}

    The attacker can identify weak, outdated or depricated tools and dependencies in the system's CI/CD pipeline via \textbf{GitHub Actions}, which is publicly availble through the \textbf{GitHub repository}.

    This would be an issue of \textit{Vulnerable and outdated components} and \textit{Software and Data integrity failures}.
    \item \textbf{\underline{Open Ports}}

    The attacker can scan the public IP addresses of the \textbf{Digital Ocean} servers to find unnecessarily or unexpectedly open ports with known vulnerabilities, which can be exploited in further attacks.

    This would be an issue of \textit{Security misconfiguration}.
    \item \textbf{\underline{SQL Injection}}

    The attacker can target the login page of the \textbf{web application} with SQL-injection attacks to strike the database.

    This would be an issue of \textit{Injection attacks}.
    \item \textbf{\underline{Exposed secrets}}

    The attacker can get access to the system or associated tools via secrets found written in the code in the public \textbf{GitHub repository}.

    This would be an issue of \textit{Identification and authentication failures} and \textit{Security misconfiguration}.
\end{enumerate}
\subsection{Risk Analysis}
\subsubsection{Likelihood Analysis}
placeholder
\subsubsection{Impact Analysis}
placeholder
\subsubsection{Risk Matrix}
placeholder
\subsubsection{Action Plan}
placeholder
