\section{Lessons Learned}

\subsection{Lesson 1: Getting Hacked}
\label{section_hacked}
After attending the lecture on security, we got hacked, leading us to experience firsthand the importance of incorporating security into a project.

The first suspicion we got was when we discovered, through our monitoring, that the response time of our server was suddenly very slow. This led us to our Digital Ocean dashboard, which showed that the server was using 100\% CPU power, which is highly unusual.

The group then scoured the server for clues as to what was happening, finding countless calls to \texttt{masscan} essentially drowning our server, as well as mysterious installations and a call to a remote script via a cronjob.

After a few failed attempts to evict the adversary it was decided to destroy the server, as we had already implemented our infrastructure as code, so provisioning and deploying a new server could be done in under ten minutes.

After some introspection into our system, we assume that the adversary got access via open ports that we were unaware of. The ports became exposed in an attempt to make the network function between servers in a Docker Swarm, which seems to override the firewall.

Learning from this, we have worked to close exposed ports from Docker and find alternative solutions to setting up the network. Another takeaway is that since we had the necessary monitoring in place to figure out that the server was being targeted, as well as having implemented infrastructure as code, we were able to detect and react to the attack fairly quickly, giving us only about one and a half hour of downtime.

\subsection{Lesson 2: Shift from Vagrant to Ansible-Pulumi}
At the beginning of the project, we chose to provision our VMs with Vagrant, inspired by the exercises from the course.
We realised later that we would eventually have to switch Digital Ocean account due to running out of credit, which in turn meant we had to streamline the setup of our VMs.

The choice of configuration management tool fell upon Ansible, which was supposed to be called by Vagrant in a config server, provisioning the web and monitoring servers.
However, it turned out that Vagrant was not the right tool when having more complicated automation and collaboration needs for our project.
After many hours of attempting to get Vagrant to work with Ansible, we found out that Vagrant was very complicated to work with in our current setup\cite{issue178-vagrant-ansible}. Upon further investigation, we discovered that Vagrant was originally designed to set up development environments, rather than for maintaining production infrastructure\cite{vagrant_vs_terraform}.

We decided to cut our losses with Vagrant and search for a more suitable tool that could help us write our infrastructure as code, where the choice fell upon Pulumi.
This taught us the importance of thoroughly researching the available tools to navigate through their advantages and drawbacks.
It was a valuable lesson to see the difference it made when taking the time to investigate different tools and their properties to make an informed decision based on the knowledge of our system's needs.

\subsection{DevOps Style}
When reflecting on how we, as a group, incorporated the style of DevOps into our way of working, we recalled the \textit{Three Ways}, which were the characterising principles for processes and behaviour in DevOps, that consists of \textit{Flow}, \textit{Feedback} and \textit{Continual Learning and Experimentation}\cite{devopshandbook}.

With the principle of flow, it was not hard to adopt the ideas of making our work visible and reducing batch sizes, as the group have previously worked in an Agile framework in other courses, which embodies those same ideas. Using a kanban board to track our tasks and their progress not only helped us stay on target, but also with confining each task, such that it could be deployed continuously\cite{devopshandbook}.

For the principle of feedback, the concept of peer reviewing via pull requests on GitHub helped install a sense of ownership over the application. Automating the process of not only testing, but also building and deploying the entire application in a continuous fashion, allowed for errors to be found and mitigated quickly compared to earlier projects we have worked on, where it was easy for the issues to pile up.

Lastly, with the principle of continual learning and experimentation, having a safe system of work\cite{devopshandbook} was crucial for us to learn and grow in a secure environment, which in turn allowed for greater experimentation in the improvement of the systems setup. By keeping weekly work logs (see appendix \ref{appendix_log}) and guides easily available, everyone had the same opportunity of better understanding of the system as a whole.
